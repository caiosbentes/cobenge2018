%--------------------------------------------%
% Template Beamer para Apresentações UFPA %
% by fiterlinge@ufpa.br              %
% Baseado em MIT Beamer Template			 %
% versao 1.1								 %
% Atualizado em 24/11/2016					 %
%--------------------------------------------%
\documentclass[unknownkeysallowed]{beamer}
% Para alterar a linguagem do documento
\usepackage[portuges]{babel}
% Para aceitar caracteres especias deretamente do teclado
\usepackage[utf8]{inputenc}
% Para seguir as normas da ABNT de citacao e referencias
\usepackage[alf]{abntex2cite}
% Para incluir figuras
\usepackage{graphicx}
% Para melhor ajuste da posisao das figuras
\usepackage{float}
% Para ajustar as dimensoes do layout da pagina
\usepackage{geometry}
% Para formatar estrutura e informacoes de formulas matematicas
\usepackage{amsmath}
% Para incluir simbolos especiais em formulas matematicas
\usepackage{amssymb}
% Para incluir links nas referencias
\usepackage{url}
% Para incluir paginas de documentos .pdf externos
\usepackage{pgfpages}
% Para ajustar o estilo dos contadores
\usepackage{enumerate}
% Para modificar a cor do texto
\usepackage{color}
% Para incluir condicoes
\usepackage{ifthen}
% Para colocar legendas em algo que nao e float
\usepackage{capt-of}
% Para definir o tema do slide
\usepackage{graphicx}
\usetheme{Berlin}
\usepackage{lipsum}
\usepackage{ragged2e}
\usepackage{etoolbox}
\usepackage{verbatim}
%eps
\usepackage{epstopdf}
%Para adicionar o Fasor
\usepackage{steinmetz}

%Para alinhar as equações a esquerda
% Para difinir cores e background
\usecolortheme{ufrn}
% Para numerar as figuras
\setbeamertemplate{caption}[numbered]

% Título
\title[]{Gamificação Como ferramenta no Ensino de Processamento Digital de Sinais}
% DatS
\date{
	\date{}}
% Autores
\author[Caio Sanches Bentes]{
Caio Sanches Bentes
\texorpdfstring{\\ SysOps Administrator at UFPA Datacenter \\
caiobentes@ufpa.br}{}
}
\institute{Congresso Brasileiro de Educação em Engenharia} % opcional
\date{\today}


\begin{document}
\frame{\titlepage}
\section[]{}
\begin{frame}{Agenda}
    \sloppy
	\tableofcontents
\end{frame}

% capitulo 1
\section{kkkkkApresentação}

    \begin{frame}{Definição}
        \begin{block}{Docker}
            O docker é uma plataforma de software que permite a criação, o teste e a implantação de aplicações rapidamente. O docker cria pacotes de software em unidades padronizadas chamadas de contêineres que têm tudo o que o software precisa para ser executado, inclusive bibliotecas, ferramentas de sistema, código e runtime. A palataforma docker é composta por algumas ferramentas:
            \begin{itemize}
                \item Docker Engine
                \item Docker Hub
                \item Docker Machine
                \item Docker Swarm
                \item Docker Compose
            \end{itemize}
        \end{block}
    \end{frame}


% Agradecimentos
\section{}
\begin{frame}{FIM}
    
    \begin{figure}
    \centering
    \includegraphics[scale=0.1]{figuras/02.png}
    \label{fig:my_label}
    \end{figure}
	
	\begin{center}
    Caio Sanches Bentes\\
    SysOps Administrator at UFPA Datacenter\\
    caiobentes@ufpa.br\\
	\end{center}

\end{frame}

\end{document}
